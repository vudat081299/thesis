\documentclass[../DoAn.tex]{subfiles}
\begin{document}

Chương này có độ dài không quá 10 trang. Nếu cần trình bày dài hơn, sinh viên đưa vào phần phụ lục. Chú ý đây là kiến thức đã có sẵn; SV sau khi tìm hiểu được thì phân tích và tóm tắt lại. Sinh viên không trình bày dài dòng, chi tiết. 

Với đồ án ứng dụng, sinh viên để tên chương là “Công nghệ sử dụng”. Trong chương này, sinh viên giới thiệu về các công nghệ, nền tảng sử dụng trong đồ án. Sinh viên cũng có thể trình bày thêm nền tảng lý thuyết nào đó nếu cần dùng tới.

Với đồ án nghiên cứu, sinh viên đổi tên chương thành “Cơ sở lý thuyết”. Khi đó, nội dung cần trình bày bao gồm: Kiến thức nền tảng, cơ sở lý thuyết, các thuật toán, phương pháp nghiên cứu, v.v.

Với từng công nghệ/nền tảng/lý thuyết được trình bày, sinh viên phải phân tích rõ công nghệ/nền tảng/lý thuyết đó dùng để để giải quyết vấn đề/yêu cầu cụ thể nào ở Chương 2. Hơn nữa, với từng vấn đề/yêu cầu, sinh viên phải liệt kê danh sách các công nghệ/hướng tiếp cận tương tự có thể dùng làm lựa chọn thay thế, rồi giải thích rõ sự lựa chọn của mình.

Lưu ý: Nội dung ĐATN phải có tính chất liên kết, liền mạch, và nhất quán. Vì vậy, các công nghệ/thuật toán trình bày trong chương này phải khớp với nội dung giới thiệu của sinh viên ở phần trước đó. 

Trong chương này, để tăng tính khoa học và độ tin cậy, sinh viên nên chỉ rõ nguồn kiến thức mình thu thập được ở tài liệu nào, đồng thời đưa tài liệu đó vào trong danh sách tài liệu tham khảo rồi tạo các tham chiếu chéo (xem hướng dẫn ở phụ lục A.7).


\end{document}