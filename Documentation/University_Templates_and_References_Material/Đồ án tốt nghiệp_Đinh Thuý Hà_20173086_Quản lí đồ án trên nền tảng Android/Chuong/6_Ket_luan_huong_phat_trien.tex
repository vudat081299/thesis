\documentclass[../Main.tex]{subfiles}
\begin{document}
\section{Kết luận}
Xuất phát từ quá trình muốn tìm hiểu và phát triển một ứng dụng liên quan đến quản lí công việc với quy mô trường đại học để quản lí đồ án, em đã thực hiện đồ án xây dựng Ứng dụng quản lí đồ án trên nền tảng Android này. 
Đồ án đã đáp ứng được các yêu cầu đề ra ban đầu. Cụ thể em đã xây dựng thành công một ứng dụng chạy trên nền tảng Android để quản lí đồ án. Thông qua việc khảo sát các ứng dụng quản lí công việc và ứng dụng eHUST từ đó em đã phân tích ra được các tính năng mà ứng dụng cần có: Người dùng là sinh viên, giảng viên có thể thực hiện các chức năng: tạo mới công việc, xem chi tiết và chỉnh sửa công việc, bình luận có thể đính kèm file vào trong công việc, có thể tải file đính kèm về, xem chi tiết đề tài( có các thông tin liên quan đến đề tài, giảng viên hướng dẫn, sinh viên thực hiện, thời gian thực hiện), ngoài ra sinh viên, giảng viên còn có thể tìm kiếm sinh viên hoặc giảng viên, tìm kiếm lớp học, xem lịch hôm nay, nhận được thông báo từ trường. Đối với người dùng là sinh viên sẽ xem được thời khoá biểu và người dùng là giảng viên có thể tạo lịch gặp với sinh viên sau khi đã có sự trao đổi về mặt thời gian giữa hai bên, nhận được các thông báo về đồ án khi sinh viên đang hướng dẫn cập nhật công việc. Người dùng là quản trị viên thực hiện chức năng phân công đồ án, quản lí danh sách phân công đồ án.

Đồ án tuy đã đạt được mục tiêu đề ra tuy nhiên do kiến thức bản thân còn hạn chế và thời gian làm đồ án là có hạn nên không thể tránh khỏi các sai sót nhất định: mỗi đề tài chỉ có một sinh viên thực hiện chưa xử lí trường hợp hai sinh viên cùng làm chung trong một đề tài, chưa thêm được thành viên vào trong đề tài. Những hạn chế của đồ án sẽ được em khắc phục và hoàn thiện trong tương lại để mang đến trải nghiệm tốt nhất với người dùng. 

\section{Hướng phát triển của đồ án}
Ứng dụng quản lí đồ án trên nền tảng Android sẽ tiếp tục xây dựng để mang lại trải nghiệm tốt hơn cho sinh viên và cán bộ trong trường với các tính năng trong tương lại đó là: cho phép giảng viên đánh giá kết quả của sinh viên cuối mỗi kì đồ án thông qua điểm số, hoàn thiện quá trình từ khi sinh viên yêu cầu làm đề tài đến khi được giảng viên chấp nhận, cho phép nhiều sinh viên làm cùng mọt đề tài và có thể thêm sinh viên đấy vào trong đồ án, cải thiện giao diện tinh tế phù hợp hơn với người dùng, quét mã qrcode để lấy được mã số sinh viên hay mã cán bộ, cho phép người quản trị viên tạo thông báo, tạo lớp học, cho phép sinh viên xem danh sách đề tài của các giảng viên và yêu cầu lên hệ thống khi đó giảng viên sẽ nhận được yêu cầu của sinh viên và đưa ra quyết định xem có cho phép sinh viên làm đề tài đấy hay không, quản trị viên dựa vào quyết định của giảng viên để đưa ra quyết định cuối về phân công giảng viên hướng dẫn cho sinh viên đó.   
\end{document}