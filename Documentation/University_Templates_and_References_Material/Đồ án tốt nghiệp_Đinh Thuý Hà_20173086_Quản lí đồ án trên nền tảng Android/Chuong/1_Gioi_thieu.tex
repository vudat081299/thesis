\documentclass[../Main.tex]{subfiles}
\begin{document}
\section{Lí do chọn đề tài}
\label{section:1.1}
Trí nhớ con người không phải là một kho lưu trữ vô hạn vì thể không thể nhớ được mình đã làm những gì, tiến độ đến đâu, còn điều gì chưa thực hiện, việc theo dõi và kiểm soát trở nên ngày càng khó khăn. Bên cạnh đó những biện phát quản lí truyền thống không thể hỗ trợ còn người thực hiện công việc một cách hiệu quả. Chính vì thế mọi người tìm đến một công cụ thông minh, tiện lợi hơn đó chính là các phần mềm, ứng dụng quản lí công việc. 

Các ứng dụng quản lí công việc ra đời mang lại nhiều lợi ích cho con người trong mọi lĩnh vực: cải thiện hiệu quả,  năng suất công việc, các công việc được sắp xếp theo thứ tự ưu tiên nhờ đó mọi người sẽ biết mình cần phải làm việc gì trước việc gì sau, nhắc nhở người dùng hoàn thành công việc trước deadline, các ứng dụng quản lí công việc còn giúp người dùng tránh sai sót, bỏ sót công việc. Có thể nói các ứng dụng quản lí công việc để trở thành một công cụ đắc lực giúp con người kiểm soát công việc, trở thành một phần không thể thiếu.

Nhằm tìm hiểu quá trình phát triển cũng như có thể tự xây dựng một ứng dụng quản lí công việc trong khuôn khổ phù hợp với sinh viên, giảng viên em đã quyết định lựa chọn đề tài xây dựng "Ứng dụng quản lí đồ án trên nền tảng Android".

\section{Mục tiêu và phạm vi đề tài}
\label{section:1.2}
Mục tiêu và phạm vi của đồ án là thiết kế và xây dựng một ứng dụng quản lí đồ trên nền tảng Android, có thể đáp ứng được các chức năng cơ bản của sinh viên và giảng viên như lên danh sách các công việc cần làm khi thực hiện đồ án, thường xuyên cập nhật tiến độ thực hiện công việc để giảng viên có thể kịp thời nắm bắt tình hình. Ngoài ra sinh viên cũng có thể xem lịch học của mình, xem danh sách lớp sinh viên, tìm kiếm một sinh viên hoặc giảng viên bất kì, tìm kiếm lớp, giảng viên có thể tạo lịch gặp mặt với sinh viên sau khi hai bên đã thống nhất về mặt thời gian. Không chỉ vậy ứng dụng cũng có không gian riêng dành cho quản trị viên có thể thống kê được số sinh viên, giảng viên, số môn đồ án, phân công  hướng dẫn đồ án.

\section{Bố cục đồ án}
\label{section:1.3}
Phần còn lại của báo cáo đồ án tốt nghiệp này được tổ chức như sau. 

Chương 2 trình bày về khảo sát và phân tích yêu cầu, phạm vi của đề tài. Trong chương này trình bày về ưu và nhược điểm của các ứng dụng đã thực hiện khảo sát. Từ đó đưa ra các chức năng cần phát triển cho ứng dụng của mình các đối tượng sử dụng ứng dụng, nêu ra các chức năng của ứng dụng, trình bày tổng quan, phân tích làm rõ các quy trình nghiệp vụ, đặc tả chi tiết các use case, các yêu cầu phi chức năng. Mỗi chức năng sẽ được mô tả thông qua biểu đồ use case phân rã, quy trình nghiệp vụ, đặc tả chi tiết cho từng use case. 

Trong Chương 3 sẽ trình bày về các công nghệ, cơ sở lý thuyết của các công nghệ, ưu nhược điểm của nó và cách áp dụng nó vào đồ án này như thế nào. 

Chương 4 trình bày chi tiết về cách phát triển và triển khai ứng dụng, từ việc thiết kế tổng quan đến việc đi sâu vào thiết kế chi tiết từng gói từng lớp, thiết kế cơ sở dữ liệu, thiết kế giao diện,... Thêm vào đó chương này cũng sẽ trình bày các thư viện và công cụ sử dụng trong quá trình phát triển ứng dụng. Sau cùng là minh hoạ một số kết quả đạt được sau khi xây dựng thành công ứng dụng quản lí đồ án và tiến hành kiểm thử.

Chương 5 đưa ra kết quả mà đồ án đạt được. Các ưu điểm và nhược điểm của ứng dụng. Chỉ ra được hướng phát triển của hệ thống trong tương lai.

\end{document}