\documentclass[../Main.tex]{subfiles}
\pagenumbering{roman}
\begin{document}

\begin{center}
    \Large{\textbf{Tóm tắt nội dung đồ án}}\\
\end{center}
\vspace{1cm}

Bộ não của con người không thể nhớ được được hết các công việc cần làm. Đấy là điều mà ai cũng từng mắc phải thế nhưng từ khi các phần mềm, ứng dụng quản lí công việc ra đời đã giúp người dùng tối ưu được hiệu suất làm việc, quản lí thời gian hiệu quả, hoàn thành công việc đúng thời hạn. Các ứng dụng này mang đến nhiều lợi ích thiết thực chính vì vậy em quyết định lựa chọn đề tài "Quản lí đồ án trên nền tảng Android" dựa trên các ứng dụng quản lí công việc phục vụ cho học tập. 

Ứng dụng quản lí đồ án được xây dựng nhằm mục đích giúp cho sinh viên, giảng viên có thể quản lí đồ án một cách hiệu quả, có thể lên danh sách các công việc cần thực hiện trong suốt quá trình làm đồ án, thông báo cho sinh viên biết các công việc sắp đến hạn để sinh viên có cách xử lí. Ngoài ra ứng dụng cũng có các tính năng tương tự tính năng của "eHust" như: sinh viên có thể xem các tin tức, thông báo đến từ trường, xem danh sách, thông tin chi tiết của các bạn trong lớp cũng như một bạn bất kì thông qua mã số sinh viên hay tên đầy đủ; xem lịch học, lịch gặp giáo viên hướng dẫn đồ án, tạo công việc; giảng viên cũng có thể tạo công việc cho sinh viên thực hiện, nhận được thông báo khi sinh viên cập nhật công việc, thống kê được số lượng đồ án đang hướng dẫn, tạo lịch gặp với sinh viên sau khi đã có sự thống nhất về thời gian với sinh viên đó; admin có thể phân công sinh viên cho giảng viên. 

\end{document}
